\section{Conclusiones}

A lo largo del desarrollo del Trabajo Práctico, a medida que se iba avanzando en cada una de las etapas, fueron apareciendo diferentes problemáticas que requirieron la modificación de alguna/as de las anteriores. Esto se debe a que al momento de cumplir con lo pedido, no se encontraba la forma de hacerlo o se lograba de manera dificultosa. Entonces se volvían a plantear partes de las etapas anteriores y hacer cambios para poder mejorarlas y cumplir con el objetivo.

El Modelo de Entidad-Relación está sujeto a constantes cambios. Por ejemplo en el caso que se deseen saber más datos, se puede volver a plantear el diagrama y modificarlo convenientemente para luego poder obtener la información necesaria de la manera más eficiente. Este cambio influirá en el diseño físico implementado en el motor de bases elegido. Otro ejemplo, tal como paso en el desarrollo de este Trabajo Práctico, es el caso que al momento de hacer alguna consulta no se consigan los datos fácilmente. Entonces se procedería de la misma manera que el caso anterior.