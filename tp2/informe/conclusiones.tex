\section{Conclusiones}

Disfrutamos resolver este trabajo práctico más que el anterior, pues pudimos implementar lo aprendido durante la cursada y básicamente dimos los primeros pasos en la creación de nuestro propio lenguaje.

Si bien es un ejemplo didáctico y bastante divertido, el propósito final del parser es difícil de probar. Es decir, que al no saber de música poder disfrutar de un \textbf{midi} creado por nosotros es muy difícil obtener una salida apreciable (ej la canción de Mario, Tiburón u otras).

Tal vez apreciaríamos más un lenguaje de programación básico, donde tenemos que tomar más desiciones implementativas y definir nuestra propia grámatica y no una tan estructurada como esta.