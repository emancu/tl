\section{Grámatica Y Tokens}

\begin{verbatim}
expression -> te co vars voices
te -> TEMPO FIGURE NUMBER
co -> COMPASS_V NUMBER DIV NUMBER
vars -> vars CONST NAME EQUAL cons_val SEMICOLON
vars ->

cons_val -> NUMBER
cons_val -> NAME
voices -> voice voices
voices ->

voice -> VOICE LPAREN cons_val RPAREN LCURLYBRACKET voice_content RCURLYBRACKET


voice_content -> compass_or_repeat voice_content
voice_content ->

compass_or_repeat -> COMPASS LCURLYBRACKET compass_content RCURLYBRACKET
compass_or_repeat -> REPEAT LPAREN cons_val RPAREN LCURLYBRACKET repeat_content RCURLYBRACKET
compass_or_repeat ->

repeat_content -> compass repeat_content 
repeat_content ->

compass -> COMPASS LCURLYBRACKET compass_content RCURLYBRACKET
compass_content -> note compass_content 
compass_content -> silence compass_content
compass_content ->

note -> NOTE LPAREN NOTE_ID COMMA cons_val COMMA figure_duration RPAREN SEMICOLON
silence -> SILENCE LPAREN figure_duration RPAREN SEMICOLON
figure_duration -> FIGURE
figure_duration -> DURATION

\end{verbatim}

\subsection{Tokens}

Las expresiones regulares de los tokens se encuentran al final del informe, en el código del archivo \textit{lexer\_rules.py}.